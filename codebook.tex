% Options for packages loaded elsewhere
\PassOptionsToPackage{unicode}{hyperref}
\PassOptionsToPackage{hyphens}{url}
%
\documentclass[
]{article}
\usepackage{lmodern}
\usepackage{amssymb,amsmath}
\usepackage{ifxetex,ifluatex}
\ifnum 0\ifxetex 1\fi\ifluatex 1\fi=0 % if pdftex
  \usepackage[T1]{fontenc}
  \usepackage[utf8]{inputenc}
  \usepackage{textcomp} % provide euro and other symbols
\else % if luatex or xetex
  \usepackage{unicode-math}
  \defaultfontfeatures{Scale=MatchLowercase}
  \defaultfontfeatures[\rmfamily]{Ligatures=TeX,Scale=1}
\fi
% Use upquote if available, for straight quotes in verbatim environments
\IfFileExists{upquote.sty}{\usepackage{upquote}}{}
\IfFileExists{microtype.sty}{% use microtype if available
  \usepackage[]{microtype}
  \UseMicrotypeSet[protrusion]{basicmath} % disable protrusion for tt fonts
}{}
\makeatletter
\@ifundefined{KOMAClassName}{% if non-KOMA class
  \IfFileExists{parskip.sty}{%
    \usepackage{parskip}
  }{% else
    \setlength{\parindent}{0pt}
    \setlength{\parskip}{6pt plus 2pt minus 1pt}}
}{% if KOMA class
  \KOMAoptions{parskip=half}}
\makeatother
\usepackage{xcolor}
\IfFileExists{xurl.sty}{\usepackage{xurl}}{} % add URL line breaks if available
\IfFileExists{bookmark.sty}{\usepackage{bookmark}}{\usepackage{hyperref}}
\hypersetup{
  pdftitle={Codebook.md},
  pdfauthor={Iris},
  hidelinks,
  pdfcreator={LaTeX via pandoc}}
\urlstyle{same} % disable monospaced font for URLs
\usepackage[margin=1in]{geometry}
\usepackage{graphicx,grffile}
\makeatletter
\def\maxwidth{\ifdim\Gin@nat@width>\linewidth\linewidth\else\Gin@nat@width\fi}
\def\maxheight{\ifdim\Gin@nat@height>\textheight\textheight\else\Gin@nat@height\fi}
\makeatother
% Scale images if necessary, so that they will not overflow the page
% margins by default, and it is still possible to overwrite the defaults
% using explicit options in \includegraphics[width, height, ...]{}
\setkeys{Gin}{width=\maxwidth,height=\maxheight,keepaspectratio}
% Set default figure placement to htbp
\makeatletter
\def\fps@figure{htbp}
\makeatother
\setlength{\emergencystretch}{3em} % prevent overfull lines
\providecommand{\tightlist}{%
  \setlength{\itemsep}{0pt}\setlength{\parskip}{0pt}}
\setcounter{secnumdepth}{-\maxdimen} % remove section numbering

\title{Codebook.md}
\author{Iris}
\date{9/21/2020}

\begin{document}
\maketitle

\hypertarget{run_analysis.r-codebook}{%
\subsection{Run\_analysis.R Codebook}\label{run_analysis.r-codebook}}

Dataset used is from:
\emph{\href{http://archive.ics.uci.edu/ml/datasets/Human+Activity+Recognition+Using+Smartphones}{Human
Activity Recognition Using Smartphones}}.

Dataset Zip File:
\emph{\href{https://d396qusza40orc.cloudfront.net/getdata\%2Fprojectfiles\%2FUCI\%20HAR\%20Dataset.zip}{UCI
HAR Dataset}}

The run\_analysis code is organized into 6 parts.

\begin{enumerate}
\def\labelenumi{\arabic{enumi}.}
\tightlist
\item
  Reading the files in
\item
  Combining files into one dataset
\item
  Extracting mean and standard deviation entries
\item
  Naming data set activities
\item
  Renaming data set variables
\item
  Creating a new tidy dataset, AvgData
\end{enumerate}

\hypertarget{reading-the-files-into-r}{%
\subsection{1. Reading the files into
R}\label{reading-the-files-into-r}}

Each file in the UCI HAR Dataset was assigned a variable and column
names \emph{features.txt}: lists each of the measurements taken. The
measurements are recorded in the order of this list

\begin{itemize}
\tightlist
\item
  features \textless- features.txt\\
\item
  columns:

  \begin{itemize}
  \tightlist
  \item
    n : the column that each measurement corresponds to in the data\\
  \item
    features : the name of the measurements
  \end{itemize}
\end{itemize}

\emph{activity\_labels.txt}: lists each of the activities the
participants do

\begin{itemize}
\tightlist
\item
  activities \textless- activity\_labels.txt\\
\item
  columns:

  \begin{itemize}
  \tightlist
  \item
    activity\_label : the numeric code for each of the activities in the
    data table\\
  \item
    activity : the name each code corresponds to
  \end{itemize}
\end{itemize}

\textbf{Data in the Test file}

\emph{subject\_test.txt}: lists which participant corresponds to each
row in the data files

\begin{itemize}
\tightlist
\item
  subject\_test \textless- subject\_test.txt.
\item
  columns:

  \begin{itemize}
  \tightlist
  \item
    participant : lists which participant corresponds to the row
  \end{itemize}
\end{itemize}

\emph{y\_test.txt}: lists which activity the participant did in each row

\begin{itemize}
\tightlist
\item
  y\_test \textless- y\_test.txt\\
\item
  columns:

  \begin{itemize}
  \tightlist
  \item
    ActivityLabel: The numeric codes for each activity observed in the
    corresponding column
  \end{itemize}
\end{itemize}

\emph{x\_test.txt}: The table of observations. Each row corresponds to a
participant in x\_subject and an activity in y\_test. Each column
corresponds to an observation in features

\begin{itemize}
\tightlist
\item
  y\_test \textless- y\_test.txt\\
\item
  columns:

  \begin{itemize}
  \tightlist
  \item
    features\$functions is the complete list of observations
    corresponding to each column of y\_test
  \end{itemize}
\end{itemize}

\textbf{Data in the train file}

\emph{subject\_train.txt}: lists which participant corresponds to each
row in the data files

\begin{itemize}
\tightlist
\item
  subject\_train \textless- subject\_train.txt.
\item
  columns:

  \begin{itemize}
  \tightlist
  \item
    participant : lists which participant corresponds to the row
  \end{itemize}
\end{itemize}

\emph{y\_train.txt}: lists which activity the participant did in each
row

\begin{itemize}
\tightlist
\item
  y\_train \textless- y\_train.txt\\
\item
  columns:

  \begin{itemize}
  \tightlist
  \item
    ActivityLabel: The numeric codes for each activity observed in the
    corresponding column
  \end{itemize}
\end{itemize}

\emph{x\_train.txt}: The table of observations. Each row corresponds to
a participant in x\_subject and an activity in y\_train. Each column
corresponds to an observation in features

\begin{itemize}
\tightlist
\item
  y\_train \textless- y\_train.txt\\
\item
  columns:

  \begin{itemize}
  \tightlist
  \item
    features\$functions is the complete list of observations
    corresponding to each column of y\_train
  \end{itemize}
\end{itemize}

\hypertarget{combining-all-the-files-into-a-single-dataset}{%
\subsection{2. Combining all the files into a single
dataset}\label{combining-all-the-files-into-a-single-dataset}}

The files in this dataset are related as such:

\begin{itemize}
\tightlist
\item
  The ``x\_{[}test\textbar train{]}'' files are the data files that
  contain the observations collected for each participant
\item
  The ``y\_{[}test\textbar train{]}'' files contain the activities
  corresponding to each row of the ``x\_{[}test\textbar train{]}'' files
\item
  The ``subject\_{[}test\textbar train{]}'' files contain the
  participant that corresponds to each row of the
  ``x\_{[}test\textbar train{]}'' and ``y\_{[}test\textbar train{]}''
  files
\item
  The ``features'' files contains the list of observations that
  corresponds to the columns of the ``x\_{[}test\textbar train{]}''
  files.
\end{itemize}

These files are the files that will be combined into one dataset. The
``activity\_label'' files translate the numeric code of the activity as
recorded in the ``y\_{[}test\textbar train{]}'' files to the activity's
descriptive name.

To combine these files, each of the {[}test\textbar train{]} pairs
(x\_{[}test{]} with x\_{[}train{]}, etc) were bound by row using rbind()
to form 3 dfs: X (with the x\_data), Y(with the y\_ data) and
participants(with the subject\_data).

Then thee participants, Y, and X files, were bound in this order, using
cbind() to make a single table called merge.

\#\#3. Extracting only the measurements on mean and standard deviation.

For this step, I used grep() on the names of the columns to identify the
columns that contain ``mean'' or ``std''' in it -- i.e.~the measurements
on mean and standard deviation. I set value = TRUE to get the character
names of the columns instead of the indices to check my work, but this
is not necessary. This was saved as selected\_names.

Then, I indexed participants, activities and selected\_names from merge,
in that order, to get a data table with the participant and activity
columns as well as all the observations on mean or standard deviation,
named selected.

\#\#4.Naming data set activities Using the data file
``activity\_label.txt'' saved as activities, I translated the numeric
codes for the activities from selected to the descriptive names by
indexing

\#\#5. Renaming the dataset with descriptive variable names

gsub() was used on the names of the data set to find and replace
undescriptive parts of the name with more descriptive counterparts, as
well as removing unnecesary symbols and periods.

\#\#6. Create a tidy dataset aggregate() was used to apply the mean()
function to each observation column of the data table, grouped by
participant and activity.

This was saved into a file called AvgData.txt

\end{document}
